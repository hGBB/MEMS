%
\documentclass[twoside,a4paper]{refart}
\usepackage[T1]{fontenc}
\usepackage{ae} 
\usepackage{makeidx}
\usepackage{ifthen}

\DeclareRobustCommand\cs[1]{\texttt{\char`\\#1}}
\def\bs{\char'134 } 
\newcommand{\zB}{z.\,B.}
\newcommand{\dH}{d.\,h.}  

\title{HamerMEMS Administration Guide}
\author{ Simon Sch\"onberger\\Korbinian Simonis (Monitoring Part)}
\date{\today}
\emergencystretch1em 
\makeindex

\setcounter{tocdepth}{2}
\settextfraction{0.7}

\begin{document}
\maketitle

\begin{abstract}
  This document describes how to install and maintain HamerMEMS.
\end{abstract}


\tableofcontents

\newpage


%%%%%%%%%%%%%%%%%%%%%%%%%%%%%%%%%%%%%%%%%%%%%%%%%%%%%%%%%%%%%%%%%%%%

\section{General}

\subsection{Structure}
The software is delivered in two main parts.
\begin{enumerate}
  \item \emph{Server}: A Web Application aRchive, that serves logic and the user interface.
  \item \emph{Monitoring Client}: A Python script that gaters all KPIs of interest on your servers you are interested in.
\end{enumerate}
\subsection{System Requirements}
In order to run the \emph{Server}, you need a Linux machine with network access running Java 11. A MySQL database must be provided on the same machine or on any machine that is reachable through the network. In order to send Alerts, a SMTP server is required. It is also recommended, that this machine is reachable from the outside via a static IP address or a Domain Name to enable simple access to the user interface.

Each machine that will be monitored must run Linux with systemd and provide a Python 3 runtime.

%%%%%%%%%%%%%%%%%%%%%%%%%%%%%%%%%%%%%%%%%%%%%%%%%%%%%%%%%%%%%%%%%%%%


\section{Installation}
\subsection{Server Deployment}
\subsubsection{Prepare the Database}
First of all, create an empty MySQL database for HamerMEMS. Please note the database server URL, the name of the database, username and password for further use.

\subsubsection{Prepare a TLS certificate keystore}
In order to secure the applications communication, you need at least a TLS private key, certificate and CA certificate. All those and other certificates that belong to the chain must be stored in a keystore of PCKS12, JKS or JCEKS format. Make sure that your end-users will be able to trust the certificate, either by retrieving it from a recognized Certificate Authority or by deploying your own CA to your client's trust stores.

\subsubsection{Generate a secret PSK}
To securely authenticate users, the server needs an unique cryptographic secret. Use the "secretgen.jar" tool from the delivery. You may run it via \texttt{java -jar /path/to/secretgen.jar}. A CLI will appear, just type in generate (see help for further options) and keep the output file.

\subsubsection{Prepare application.properties}
An example application.properties file is provided in the delivery. Load the file into a plain text editor and replace the example values using the information from the former steps. Further options are extensively described in the file itself and need attention, too.

\subsubsection{Copy files to the server}
Copy application.properties, the secret, the keystore and server.war to the machine you would like to deploy on.

\subsubsection{Run the server}
To run the server application, you simply need to run it on the JVM via \texttt{java -jar /path/to/server.war}. You may enable it as a systemd unit in order to keep it running even after reboots etc.

\subsection{Monitoring Client Installation}
\subsubsection{Preparations}
Copy the mc\_hamer.zip file to the node you want to integrate to the EMS.
E.g. \texttt{scp mc\_hamer.zip <username>@<ip\_of\_node>:<location>}

Unzip the package: \texttt{unzip mc\_hamer.zip}

\subsubsection{Automated install}
Fully automated with the install.sh script:\\
\texttt{bash install.sh -u <ip\_of\_backend>:<port> -t <secret\_pin> \\((optional:)) -i <send\_intervall> -n <name>}\\
-u and -t are required, for the send\_interval and name default values will be used if not specified.

The secret PIN can be requested via the applications user interface. For this, log in with an administative account, navigate to "Node Management" and click on the big "+" symbol.

\end{document}
